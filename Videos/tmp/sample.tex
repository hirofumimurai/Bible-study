\documentclass[twocolumn]{jsarticle}
\begin{document}


「平和をつくる」地区大会の
3日目午前の部に

ようこそおいでくださいました

まずミュージックビデオを
楽しみましょう

神との平和な関係を築くなら
幸せになれる ということが分かります

どうぞお楽しみください

兄弟姉妹 こんにちは

3日目の主題は
ローマ 15:13に基づいています

「希望を与える神が⋯⋯皆さんを

あらゆる喜びと平和で
満たしてくださ[る]」というものです

ではご一緒に 101番の賛美の歌

「一致して共に働く」を歌いましょう

歌は101番です

私たちは時折

平和をつくるのがとても難しいと
感じることがあります

どんなときでしょうか
どう対処できるでしょうか

そうした点を 7部から成る
シンポジウムで考えます

平和の種をまいて平和を
刈り取った人たちの手本に

どのように倣えるかを学びます

まず 奉仕委員会の援助者

ウィリアム・ターナー兄弟の
話に耳を傾けましょう

主題は

「平和の種をまいて
平和を刈り取った人たち

ヨセフと兄たち」です

聖書には クリスチャンが
生活の指針にできる

基本的な原則が
たくさん収められています

例えば 格言 13:20には

「賢い人たちと共に歩むと
賢くな[る]」とあります

格言 28:20によると

「忠実な人は多くの祝福を受け」ます

皆さんも大切にしている原則が
あることでしょう

では 平和をつくることに関しては
どうでしょうか

1つの重要な原則を
思いに留めている必要があります

どんな原則でしょうか

ご一緒にガラテア 6章を開きましょう

注目したいのは ガラテア 6章の

7節と8節です

こうあります

「思い違いをしてはなりません

神はご自分を侮る者を
大目に見ることはありません

人は自分がまいているものを
必ず刈り取ることになります

罪深い欲望のままにまいている人は

罪深い欲望によって腐敗を刈り取り

聖なる力に導かれてまいている人は

聖なる力によって永遠の命を
刈り取ることになるのです」

原則は まいたものを
刈り取るということです

農業をする人は

当然 自分が刈り取りたいものの
種をまきます

そして すぐには実がならないことを
理解しています

でもいずれ実がなることを確信して
その時を待ちます

平和もそれと似ています

平和を刈り取るには
平和の種をまかなければなりません

そして いずれ平和を
刈り取れることを確信し

努力しながらその時を待ちます

このシンポジウムでは
古代や現代のエホバの民の例を考えます

問題に直面しながらも

平和の種をまいて
刈り取った人たちの例です

さて 平和を乱すものの1つに
家族間の問題があります

不完全さのせいで

家族との関係が緊張するということが
あるかもしれません

互いを傷つけるようなことを言ったり
したりすることもあります

でも大抵はささいなことなので

すぐに平和の種をまいて
刈り取れるでしょう

でも もっと深刻な場合は
どうでしょうか

家族が言ったりしたりしたことのせいで
深く傷ついたとします

そういう時でも 平和の種をまいて

平和を刈り取ることは
不可能ではありません

聖書に出てくるヨセフと兄たちの例から
そのことを考えてみましょう

よく知っている話ですね

兄たちはヨセフをねたんで
奴隷として売りました

これはささいな過ちでは
ありませんでした

ヨセフはエジプトに連れていかれます

そして後に無実の罪で
牢屋に入れられます

それで 13年もの間

ヨセフはひどい目に遭いました

何も悪いことをしていないのにです

ヨセフは次のように考えたでしょうか

「また兄たちに会えたとしても

お互いの平和のために

もう一切関わりを
持たないようにしよう」

そう考えたとしても無理はありません

でもヨセフは違いました

それからしばらくたった後

ヨセフも兄たちも平和の種をまきます

どうしてそうできたのか
詩編 105編を見てみましょう

詩編 105編の19節を読みましょう

特に前半の言葉に注目したいと思います

「エホバの言葉が彼を磨き上げた」

ヨセフはエホバに磨き上げていただき

エホバに倣った考え方や
振る舞い方を心掛けました

憤りを抱きませんでした

ヨセフが兄たちに
ひどい目に遭わされたと

誰かに話したというようなことは

聖書のどこにも書かれていません

ファラオにさえ

自分が兄たちに
奴隷として売られたということは

話さなかったようです

兄たちはどうでしたか

何年もたってから
エジプトでヨセフに会います

ヨセフはわざと末の弟である
ベニヤミンをひいきして

兄たちを試しました

でも兄たちは変化していました

人をねたまなくなっていたのです

弟や父親ヤコブのことを
深く気遣っていました

ヨセフは兄たちが悔い改めたのを見て

許しました

ヨセフも兄たちも

平和の種をまくように努力したので

とても良い結果を
刈り取ることができました

また家族として
仲良く暮らすことができたのです

このヨセフと兄たちの例から
何を学べるでしょうか

次のビデオの中で

どのように平和の種が
まかれたかに注目しましょう

ジェレミーは毎朝 本当に毎朝

すっごく大きな声で歌ってたんだ

いや 僕だけじゃないよ

そうだけど でも一番大きかったよ

兄さん いつもそういうこと
ばっかり言う

2人は本当にいつも仲がいいですよね

うん でも

そうじゃないときもあった

うん でも もともと仲は良かったんだ

兄と弟というよりは

友達だった

うちは家族で仕事をしてたんだけど

問題が起きて
ささいなことから口論になった

おまえはうちの仕事の流れが
全く分かってないんだ!

批判ばっかりしないで
一度くらいこっちのことも・・・

ジェレミー!

これ どれだけ
コストかかるか分かってんのか?

しょうがないだろ
こっちは聞いてなかったんだから

足引っ張んなよ!

引っ張ってるのは兄さんだろ!

仲良くやろうと努力した

でも 悪くなる一方だった

2人の間に厚い壁を感じた

人前では普通に振る舞っていた

でも 心の中では全然許していなかった

自分が偽善者のように感じた

このままじゃいけないと思った

エホバの助けがどうしても必要だった

心から許して

ニックと仲直りするために

傷が癒えるには時間がかかった

以前のように
信頼し合えるようになるにも

でも仲直りできた

平和な関係を取り戻せて本当に良かった

世界が大変なことになる前に

では どうすればヨセフと
兄たちに倣えるでしょうか

3つの点を考えましょう

最初の点は
エフェソス 4章の32節にあります

読んでみましょう

エフェソス 4:32

「親切な人になり

温かい思いやりを示し合い

神がキリストによって
寛大に許してくださったように

寛大に許し合いましょう」

1つ目に 人を寛大に
許す必要があります

もし憤りや恨みの気持ちを
募らせてしまうなら

許すのが難しくなることでしょう

ビデオの中で ニックとジェレミーは
ささいなことがきっかけで口論になり

深刻な仲たがいへと発展しました

なぜでしょうか

ジェレミーが兄のニックに対して
憤りを抱き続けたからです

エホバが寛大に許してくださるように

私たちも人を許さなければなりません

2つ目の点は

いつまでも根に持たないことです

格言 17:9には

「くどくど言う人は親友を引き離す」と
書かれています

ジェレミーもそういう
状態になっていましたね

兄との間に壁を感じていました

でも 平和の種をまくように
努力しました

兄と腹を割って話し合うようにし

共に時間を過ごし
一緒に奉仕もしました

それでやっと2人の仲が改善されました

とはいえ

すぐに元通りにはなりませんでしたね

2人が心の傷を癒やして

再び信頼し合うには

かなりの時間が必要でした

すぐに成果を期待せず

じっくり取り組むことが大切です

農業をする人のように

実がなるのを辛抱強く待って
努力するなら

いずれ平和を刈り取ることが
できるでしょう

では3つ目のとても重要な点です

憤りの気持ちを
捨て去ることができるように

エホバに助けを祈り求めます

ヨセフのことを思い出してください

詩編 105:19にあったように

「エホバの言葉」に助けられ
磨き上げられたので

憤りを捨て去ることができました

ビデオの中で ジェレミーも
エホバの助けのおかげで

ニックを心から許して
仲直りすることができました

エホバは私たちも助けてくれます

家族との平和な関係が
損なわれると つらいものです

でも まいたものを刈り取るという
原則を思いに留めて

平和の種をまきましょう

先ほど考えたように 寛大に許し

いつまでも根に持たないようにし

エホバに助けを求めるなら

その努力は祝福されます

そしてきっと 平和を刈り取って

家族で仲良く
幸せに暮らせることでしょう

次に 教育委員会の援助者
ロナルド・カーザン兄弟が

「平和の種をまいて
平和を刈り取った人たち

ギベオンの人たち」という話をします

はるか昔から今に至るまでずっと

ある対立が続いてきました

どんな対立でしょうか

不従順な人間たちが

エホバの意志に逆らってきたのです

多くの人たちがエホバの
望まれることを行おうとせず

エホバと和解することを拒んで

平和を壊してきました

一方で 歴史を通じて大勢が

喜んでエホバの望まれることを行い

エホバの考えに合わせてきました

その人たちは平和をつくってきたのです

その人たちは平和の種をまいて

エホバとの親しい平和な関係を
刈り取ってきました

新しい平和な世界で
永遠に生きる希望も持っています

とはいえ 時として エホバからの
指示を受け入れて従うことを

難しく感じることがあるかもしれません

それにはエホバの組織からの
指示も含まれます

指示の理由がよく分からなかったり

自分の考えと違ったりすると

特に難しく感じるものです

平和を刈り取るには何が必要でしょうか

謙遜さです

謙遜さとは 自分を低く見ることであり

傲慢さが全くないことです

謙遜な人は 自分の
したいようにしようとはせず

エホバや他の人たちと
平和な関係でいることを優先します

いつも平和をつくるようにするのです

謙遜に平和の種をまくと
どんな結果になるでしょうか

格言 22:4の言葉に注目しましょう

どんなものを刈り取れるでしょうか

「謙遜さとエホバへの
畏れがもたらすのは

富と栄光と命である」

そう 永遠の命です

聖書には 謙遜に
エホバの意志に従うことによって

平和の種をまいた人たちの例が
たくさん出てきます

ギベオンの人たちの例を考えましょう

彼らは最初 イスラエル人の
敵ではなかったでしょうか

はい

では どのように謙遜に平和の種を
まいて平和を刈り取ったのでしょうか

ヨシュアが率いるイスラエル人が

ヨルダン川を渡って約束の地に入った後

エホバは彼らにカナンの国々を
滅ぼすようにと指示しました

当然 ギベオンの人たちも
含まれていました

ヨシュア 9章から分かるように
ギベオンの人たちは

カナン人を全て滅ぼすことが
エホバの意志であることを

はっきりと知らされていました

イスラエルがエリコとアイを
打ち負かしたことからも

エホバがイスラエル人のために
戦っていることは明らかでした

ギベオンの人たちは恐れました

でも 滅びることがエホバの意志なら
何ができるでしょうか

ある名案を思い付きました

代表者たちを送って イスラエル人と
エホバと和平を結ぶことにしたのです

9:11にはこんな言葉があります

「私どもはあなた方に仕えます

どうか私どもと契約を結んでください」

これが和平を求める言葉でした

4節には 彼らが「抜け目ない
行動を取った」とあります

ぼろぼろの服や持ち物で

イスラエル人たちを
だまそうとしたのです

実際には30キロほどの距離でしたが

非常に遠い土地から来たように
見せ掛けました

うまくいったでしょうか

だますのは良くないことでしたが
計画通りにいきました

15節にあるように
ヨシュアはギベオンの人たちと

平和の契約を結びました

このことはエホバの意志に
沿っていたと言えます

そして ギベオンの人たちの謙遜さと

平和を求める態度の表れでした

ギベオンの人たちは
与えられた指示に従うことによって

エホバの意志やエホバの代表者である

ヨシュアに従う謙遜さを表しました

27節にはこう書かれています

「その日 ヨシュアは彼らを

民およびエホバの祭壇のために

まきを集める者 水をくむ者とした」

ギベオンの男性たちは
腕の立つ戦士だったと思われますが

清い崇拝をサポートするために

どんな仕事でも喜んで行いました

謙遜に指示に従ったので

命が守られ 平和を味わいました

どうすればギベオンの人たちに
倣えるでしょうか

エホバの意志に沿った生き方を
することによってです

かなり大きな変化が
必要な場合もあります

考え方を変えたり 悪い習慣をやめたり

良くない交友を避けたりします

ほかにもギベオンの人たちに倣える点は

清い崇拝をサポートするために

与えられるどんな割り当ても
喜んで果たすことです

何が分かりましたか
平和を刈り取るには

謙遜に指示に従う
必要がある ということです

そのことが次のビデオの中で
強調されています

エホバの指示に従うなら
良い結果になります

そうだった

パンデミックがあんなに長く続くとは
誰も思わなかった

いろいろなニュースが飛び交って

何を信じていいか分からなかった

妹が失業してうちに引っ越してきた

それで 私とケリー 妹 父の
4人で暮らすことになった

20秒だぞ 手洗いは20秒

行ってくんね

うん

あ マスクは?

要らない

これ これ使って

ちゃんとディスタンス取るから

兄さん?

仕事場では誰も
マスクなんか着けてないんだ

会社で何人感染したか知ってる?
ゼロだよ

今はまだね この間の
統治体からの話 見ただろ

警戒を緩めちゃいけない 油断するな

そんなの分かってるよ

そりゃ 僕たちを守ろうとしてくれてる
っていうのはうれしいけど でも・・・

でも何?

でも ちょっと兄弟たち
深刻に考え過ぎじゃない?

でももちろん 統治体の兄弟たちは
間違っていなかった

私が間違っていた

1週間後 グループの若い兄弟が

コロナで入院した

兄弟はしっかり感染対策をしていたのに

自分が油断していたことに気付いた

今は警戒を緩めるべき時ではありません

油断することなく

用心し続けましょう

「うちの家族は大丈夫」とは
考えないでください

エホバの組織が私たちを
本当に守ろうとしていたこと

自分の決定が他の人に
大きな影響を与えること

家族にも影響を与えることが分かった

パンデミックが続く中

謙遜に従うことがいかに大切かを学んだ

エホバからの指示に従うことで

安全に暮らせるだけでなく

平和でいられる

タイムリーな内容でしたね

今の状況にぴったり
当てはまっていました

ビデオに出てきた兄弟のように

私たちも自分の意見や好みを
重視するのではなく

エホバの指示に謙遜に従うために

考え方を調整する必要が
あるかもしれません

もちろん エホバを代表している

「忠実で思慮深い奴隷」の指示に
ついても同じことが言えます

大切なことを学べました

謙遜に指示に従うなら

家族が安全に暮らすのに
役立つだけでなく

幸福や平和を味わえます

ぜひ覚えておきたい点ですね

いつも謙遜に指示に従う上で

どんなことが助けになるでしょうか

エフェソス 5:17で

使徒パウロがどんなことを
述べているかに注目しましょう

「もう無分別なことをしてはなりません

いつでも エホバが何を
望んでいるかを見極めましょう」

聖書を読むと エホバが
望んでいることが分かります

でも エホバの考えが
はっきり分からない時や

自分の状況にどう当てはめたら
いいかがよく分からない時は

どうしたらいいでしょうか

パウロによれば「エホバが何を望んで
いるかを見極め」ることが大切です

何ができますか
貴重な宝石を見つけるために

地面の下を深く掘っていく
必要があるように

聖書を表面的に読むだけで
満足してはなりません

聖書を深く掘り下げて調べるようにし

宝石のような聖書の原則を
見つける努力を払いましょう

そして 祈りながら自分に
どう当てはまるかを考えます

そうすると エホバと同じような
考え方ができるようになっていきます

エホバの考えを見極めるようにすれば

自分の考えに固執しなくなるはずです

エホバの指示や統治体からの指示にも

喜んで従おうという気持ちになります

ギベオンの人たちのように
平和の種をまき

エホバやエホバの民と
平和な関係でいられるようにしましょう

謙遜にエホバの意志に従うなら

平和を刈り取っていつまでも
幸せに暮らせるでしょう

次に 教育委員会の援助者
ケニス・フローディン兄弟が

「平和の種をまいて
平和を刈り取った人たち

ギデオン」という話をします

これからギデオンの例について
考えたいと思います

この話ではどんな問題を
取り上げると思いますか

次の点です

仲間のクリスチャンとの間の不和です

不和などあるんでしょうか

イエスは 自分の弟子たちは愛によって
見分けられると言ったのではないですか

もちろんそうです

クリスチャンにとって愛は
一番大切であり

私たちはその愛を感じます

でも 私たちはいわば
不完全さに染まっています

そのせいで不和が
生じてしまうことがあるのです

ヤコブ 3:2は率直に

「私たちは皆 何度も
過ちを犯します」と言っています

毎日とかしょっちゅうではないとしても
何度も過ちを犯してしまうのです

誰かを傷つけてしまうかもしれません

感情を害するようなことを
言ってしまったり

逆に言われたりします

大抵はわざとではなく
単なる誤解だったりします

悪気はなく 何気なく言った言葉が

相手の気に障る場合もあります

自分が言ったことやしたことに対して

思いがけない反応が
返ってくる場合もあります

いずれにしても 不和が生じたなら

平和の種をまいて
平和を刈り取らなければなりません

種をまいてから実を刈り取るまで
あまりかからない場合もあります

このシンポジウムの
最初の話でも考えた通り

農業をする人のように 実がなることを
確信して種をまきましょう

仲間のクリスチャンとの間に
不和が生じた場合

ギデオンの例について考えると

平和の種をまいて
平和を刈り取るのに役立ちます

ギデオンは
ミディアン人と戦っていた時

仲間であるエフライムの人たちに
追撃するよう頼みました

「裁き人の書」にある通りです

戦いは大勝利に終わりました

ところが エフライムの人たちは感情を
害して ギデオンに詰め寄りました

もっと早く戦いに呼んでほしかった と
不満をぶつけたのです

裁き人 8:1

「エフライムの人たちは
ギデオンに言った

『一体どういうつもりだ
ミディアンと戦う時に

なぜわれわれを呼ばなかったんだ』

そしてひどく文句を言った」

この人たちは武装した兵士でした

もし不和が解決されなければ
大変なことになっていたかもしれません

もしかしたら殺し合いに発展していた
可能性もあります

ではこの時 ギデオンは
どのように平和の種をまいたでしょうか

2, 3節を読みましょう

「ギデオンは言った
『皆さんがしたことに比べれば

私は大したことはしていません

エフライムのブドウの収穫の残りは

アビ・エゼルの収穫に
勝っているではありませんか

神はミディアンの高官オレブと
ゼエブを皆さんの手に渡しました

皆さんがしたことに比べれば
私は大したことはしていません』

[次です]ギデオンがこう話すと

エフライムの人たちの
気持ちは治まった」

この時は 割とすぐに
平和を刈り取れました

ギデオンは よく考えた言葉で
平和の種をまき

1回の会話で
平和を刈り取ることができました

次のビデオの中で ある兄弟が

誤解が深刻な不和に発展しないために
どうするかに注目しましょう

エホバの証人ってすごいですよね

何かが起きると 兄弟たちは
すぐ助けに行きますよね

そういえば あの時も
仕事がたくさんあった・・・

援助を申し出る人がいつもいた

でも 働きたい人が多かったので

自分が外されているんじゃないかと
思う人が出てきた

あ レイ どうしたの?

マシュー 君が会衆の食料分配を
担当してるんだって?

うん 何とかやってる

何か手伝える?
僕のトラック いつでも使えるよ

今日の荷物は全部
積み終わったから 大丈夫かも

でも声を掛けてくれてほんとありがとう

そう じゃあ 次配るのはいつ?

実は これから出るところなんだ
後でかけ直してもいい?

ああ うん 分かった

レイは本当によく働く兄弟で

僕たちの知らないところでも
よく動いていた

どんな奉仕も一生懸命やっていた

誰だったの?

「奉仕者は十分集まった」って伝えたら

思ってもない反応が返ってきた

レイ ミュートじゃない?

ということは 僕には奉仕に
参加する資格がないってこと?

いや そうじゃなくって

お母さんのお世話 一生懸命
してるでしょ だからレイには・・・

ポールも今回入ってなかった?

うん そうだけど

あとオリバーは?

オリバーは補佐を
やってくれているけど・・・

どうやって選んでいるの?マシュー

友達ばっかり選んで
ほかの人は入れないんだ

兄弟たちが僕のことを
無視しているとしか思えないんだけど

レイ

ほんと悪かった

ごめんね

最初からきちんと話しておけばよかった

レイはお母さんのお世話
一生懸命してるよね

これ以上負担増やしたくないと思って

それもあるんだけど

レイとお母さんを
感染のリスクにさらすようなことを

お願いしたくなかったんだ

そうだね それは僕も心配してる

そうだよね 万一

お母さんに何かあったら大変だし

姉妹は会衆の宝だから

ありがとう

レイも大切だけどね

いや ほんとに
レイはみんなの手本だよ

エホバはレイが会衆やお母さんのために
していることをよく見ている

そして喜んでいるよ

ありがとう すごくうれしい

もし ちょっとでも
できることがあったら

リモートでもできることなら
喜んでやるよ

あ それはいいね

じゃあ 早速何かお願いしようかな

いかがでしたか

2人ともいい兄弟でしたが

不完全さのせいで同じ状況を
別々の観点から見ていました

マシューはレイのためを思って
仕事を頼みませんでした

レイの母親が病気だったので
配慮したのです

でも レイは腹を立てました

マシューが友達ばかりを選んで自分を
無視している と言って責めました

では どう解決されましたか

まず 幸いなことに

レイとマシューは2人きりで話して

レイは気持ちを伝えることができました

その結果どうなったでしょうか

マシューはレイの話をよく聞いて

レイの気持ちを
理解することができました

本当に良かったですね

それからどのように会話が
続いていきましたか

マシューはレイの気持ちが分かった後
謝っていましたね

ギデオンのようによく考えられた言葉を
言ってレイを落ち着かせ

平和の種をまくことができました

1回の会話で平和を刈り取れました

では 今後仲間との間に
不和が生じた場合

どのようにギデオンに倣えるでしょうか

テモテ第二に鍵があります

エフライムの人たちのことを
思い起こすと

ギデオンに対してかなりけんか腰でした

剣を振りかざしているかのようでした

でもパウロの言葉に

ギデオンが示した態度が
よく言い表されています

テモテ第二 2:24

「主の奴隷は争う必要はありません
[言葉でも剣でもです]

必要なのは [ギデオンのように]
誰にでも穏やかに接すること

[そして大事なのは ]不当な扱いを
受けても自分を抑えること」です

簡単ではありませんが 「不当な扱いを
受けても自分を抑える」なら

平和の種をまくことができます

相手に落ち度があっても
それがささいな問題なら

あえて取り上げないことに
できるかもしれません

でもマシューの場合
傷ついたレイに感情をぶつけられました

マシューは相手の観点が違うことに
気付きましたが

謙遜に「最初からきちんと
話しておけばよかった」と言いました

そのようにして平和の種を
まくことができたのです

さて ギデオンは
1世紀よりずっと昔の人でしたが

後にクリスチャンたちに与えられた
大切な教えに従っていたと言えます

フィリピ 2:3にあるように

「自分より他の人の方が
上だと考え」ました

ギデオンは何と言っていたでしょうか

2回も「私は大したことは
していません」と言いました

その謙遜な態度により

エフライムの人たちの気持ちは
治まりました

どう倣えますか

たとえ自分には落ち度がないように
思えたとしても

不和が生じてしまったことについて
謙遜に謝れます

相手のために祈ったり
誠実に褒めたり 親切にしたりできます

そして 話し合う場合には
快い言葉を優しく語るようにしましょう

クリスチャンの仲間との間に
不和が生じても

素早く解決できるなら
素晴らしいことです

でも その時の状況や

個性の違いや 問題の性質によって

うまくいかないこともあります

それでも 私たちは
平和の種をまくために

自分にできることを行いましょう

そして いずれ平和を刈り取れることを
確信して 待ちましょう

そうすれば

素晴らしい平和の実を
存分に味わうことができるでしょう

出版委員会の援助者
ロバート・ルシオーニ兄弟が

このシンポジウムの次の話をします

主題は 「平和の種をまいて
平和を刈り取った人たち アビガイル」

平和をつくるのは簡単ではありません

私たちは不完全ですし

いろいろな問題に直面するので

普段でも平和をつくるには
努力が必要です

でも もっと大変なのは

1人で奮闘しなければならない場合です

家庭や会衆で 自分は
平和をつくりたいのに

相手にその気がない場合
難しく感じることでしょう

例えば 会衆内にいる自分の友達が

別の友達と仲たがいしているものの

仲直りしようとしないかもしれません

あるいは 家庭内で配偶者が
聖書の教えに従おうとしないため

平和をつくるのが難しく
感じる場合があります

どうしたらいいでしょうか

そういう場合に 聖書のアビガイルの
手本を参考にできるかもしれません

サムエル第一 25章を開いて

アビガイルと夫のナバルについて
書かれていることを考えてみましょう

3節には ナバルが「荒っぽ[い]」
人だったと書かれています

25節では アビガイルが
ナバルのことを

「分別がない」と言っています

17節では ナバルの召し使いが

「ご主人はどうしようもない方で
誰も何も言えない」と言っています

こういう性格の人でしたから

ダビデの部下たちにひどい態度を
取ったのもうなずけます

ダビデは復讐心に燃えます

こういう状況でどうすれば
平和をつくれるでしょうか

アビガイルはどうしましたか

考えてみると どうすることも
できたでしょうか

何もしないという
選択肢もあったでしょう

ナバルと一緒に暮らすのは
大変だったに違いありません

それで ただ黙ってダビデの行動を
見守ることもできたでしょう

あるいは ダビデに会いに行って
火に油を注ぐこともできました

ナバルのせいで自分も
ひどい目に遭ってきた と

話すこともできたでしょう

そのようにしていたら自分にとって
楽だったかもしれませんが

そうしませんでした

アビガイルは 夫の行動を
変えることはできませんでしたが

悲惨な争いに発展しないように

自分にできることをしました

サムエル第一 25章に戻って

アビガイルの言葉に注目してみましょう

27-31節を読みたいと思います

「あなたのために持ってまいりました
この贈り物を

あなたに従う部下たちに
お与えください

どうか私の違反をお許しください

あなたはエホバの戦いを
戦っておられるのですから

エホバは必ずあなたの家系を
存続させてくださいます

あなたはこれまでずっと何の
悪いこともしてこられませんでした

誰かがあなたを追跡して命を狙う時

あなたの命はエホバ神の命の袋の中に
安全に包まれます

一方 神はあなたの敵の命を

石投げ器で投げる石のように
放り投げます

エホバが約束通りあなたに
さまざまな良いことをしてくださり

あなたをイスラエルの
指導者に任命する時

あなたは 理由もなく人の血を流して

自分の手で復讐したという後悔の
気持ちを抱くことはありません

エホバがあなたに良くしてくださる時

どうか私を思い出してください」

アビガイルはどうしたでしょうか

贈り物を持ってきて
ダビデの前でひれ伏し 

「悪いのはこの私です」と言ってから

神の考えに沿ったことを話して
ダビデに訴え掛けました

アビガイルはダビデの気持ちを静め

災難を回避できました

まさに平和をつくったのです

アビガイルは自分のことよりも

エホバのお名前のために
どう行動するのが最善かを考えました

それは簡単ではありません

でもどうなりましたか

平和の種をまいたので
平和を刈り取ることができました

この手本にどのように倣えるでしょうか

倣うことはなぜ大切でしょうか

次のビデオを見て 考えてみましょう

その後 状況が難しくなって

平和に仲良くやっていくことが
試されたんですよ

でも ティムは みんなが前向きで
いられるよう助けてくれましたね

その点は エイミーから
いろいろ学びました

みんなが仲良くなれるよう
よく助けていたんです

例えば コロナが始まる前

2人の開拓者の間で
ちょっとしたことがあったんです

ベッキーは 元気によく働き

物おじせずに伝道する人

レナは経験もあり 人のためによく動き

教えるのが上手だった

お互い受け入れられないところが
あったみたいです

もう行ける?

大丈夫?

はあ ベッキーったら 本当に
形だけの開拓者なんだから

全然奉仕に出てこない

こんなこと言っちゃいけないけど
もっとちゃんと奉仕した方がいいと思う

ベッキー 大都市の
公共エリアやってるし・・・

うんうん知ってる で LDCもでしょ

確か 設備だっけ?

基礎工事 ベッキー体力あるからね

私たちも前はそうだったよね

ベッキーほどじゃないと思うけどね

そんなことない 今のベッキー
なんてもんじゃなかったじゃない

レナのペースに付いていけなかった
あの赤い車で一日中奉仕したよね

すごい前のことに感じる

そうね

あのね ベッキー見てると
レナみたいって思う

ていうか 私たち 20年前の

ベッキーはよく働くし

エホバへの奉仕を一生懸命やってる

レナみたいにね

今度一緒にお茶しない?
ベッキーも呼んで

もっと知り合えるかもよ

次の週 2人を家に呼んだんです

2人とも お互いのことを
よく知るにつれて

仲良くなれたんですよ

「賢い人たちの舌は
人を癒やす」という格言

エイミーはその言葉を実践していました

よし みんな集合!

急に状況が変わって

同じ刑務所に入れられた時

みんなが一致するために

いつも平和を心掛けた

エホバの助けで

そうできた

エイミーにはどんな選択肢が
あったと思いますか

何もせず レナがベッキーに対して

ネガティブな感情を抱くままに
することもできたかもしれません

あるいは レナに同意したり

自分もベッキーに
いら立っていると言ったりして

火に油を注ぐことも
できたかもしれません

でもそうしませんでした

レナがベッキーのことを違う観点で
見ることができるように助けて

平和の種をまいたのです

その結果 2人は仲良くなり

平和を刈り取ることができました

私たちも ぜひエイミーと同じような
行動を取りたいと思います

その点で使徒パウロが何と
言っているかに注目してみましょう

ローマ 12:18です

「できる限りのことをして

どんな人とも平和な関係で
いるようにしましょう」

「できる限りのことをして」とあります

問題が深刻になる前に

できる限り平和の種をまくなら

会衆や家庭の平和に貢献できます

例えば 家の中で何かが
燃えていたらどうしますか

そのうち消えると考えて無視しますか
油を掛けますか

いえ もちろんすぐに
消そうとしますよね

そうしなければ自分の家が
大変なことになってしまうと

分かっているからです

平和を乱すような問題が起きたときにも

同じことが当てはまります

何もせずに問題を放置したり

悪化させたりするなら
どうなるでしょうか

その問題のせいで 家庭や会衆が
大変なことになってしまうでしょう

では どうすれば平和の種を
まくことができるでしょうか

アビガイルの手本を
思い起こしてみましょう

アビガイルは どうするのが最善かを
よく考えて 行動しました

問題を放置したりしませんでした

敬意を込めて ダビデの
気持ちを静めました

火に油を注ぐようなことに
ならないように注意しました

そうした点に倣いましょう

ほかにもどんなことができるでしょうか

聖書やエホバの証人の
出版物を調べるなら 

どうすれば平和の種を
まけるかがさらに分かります

長老にアドバイスを
求めることもできます

平和をつくるのが特に大変なのは
どんな場合でしょうか

アビガイルのように 家庭内の
平和が乱されている場合です

とてもつらいですね

そういう状況でも

配偶者の良いところに目を
向けるようにしましょう

気に入らないところばかり
見ないようにします

さて 現実的な見方も必要です

使徒パウロが何と言っていたか
覚えていますか

ローマ 12:18で

「できる限りのことをして」
と言っていました

「できる限りのことをして」も

平和をつくれない場合もある
ということです

そういう場合 自分の
ベストを尽くしたら

後はエホバに委ねるようにしましょう

では 平和をつくることを
いつも意識するようにして

努力を続けていきましょう

会衆内でも家庭内でも

私たちが平和の種をまく努力をするなら

きっと平和を刈り取れるでしょう

たとえ相手がなかなか応じなくてもです

エホバとの平和と心の平和を味わい

満たされた気持ちになるからです

それでは 今後いろいろな
問題が生じたとしても

ぜひアビガイルのように

平和をつくる人になりましょう

次に 教育委員会の援助者
ウィリアム・マレンフォント兄弟が

「平和の種をまいて
平和を刈り取った人たち

メピボセテ」という話をします

私たちは不完全です

残念ですが それが現実です

不完全ですから 誰でも間違いを
してしまうことがあります

それで 不公平な扱いを
受けることもあります

仲間のクリスチャンからもです

めったにないことですが

中傷や詐欺の被害に遭うことさえ
あるかもしれません

聖書に出てくる中傷の例を考えましょう

それはダビデ王の時代に
イスラエルで起きたことです

メピボセテという名の人が

ツィバという名の人から
中傷を受けました

その時にメピボセテがどう行動したかに
注目したいと思います

そうすると 私たちが平和の種をまいて

平和を刈り取るのに役立ちます

メピボセテの父親は
ダビデの親友ヨナタンで

祖父はサウル王でした

ヨナタンとサウルが死んだ時

ダビデ王は残されたメピボセテに

祖父の土地を全て与えました

そしてツィバを
メピボセテの召し使いにし

メピボセテに与えた土地を
管理させることにしました

しばらくして ダビデの息子が
謀反を起こしたため

ダビデはエルサレムから逃げました

その時ツィバがダビデに会いに行きます

ダビデがツィバに
どうしてメピボセテは

一緒に来なかったのかと尋ねると

ツィバはメピボセテを中傷しました

全く事実ではありませんでしたが

「メピボセテは自分が王になることを
望んでいる」と言ったのです

残念なことに

ダビデはツィバのうそをうのみにし

メピボセテの土地を
ツィバに与えてしまいました

しばらく後に メピボセテは
ダビデ王に会って

自分がダビデと一緒に行かなかった
理由について説明します

そして ツィバがうそを
ついたことを伝えます

ダビデはメピボセテの話を聞いたものの

土地を元には戻さず

「ツィバと分け合いなさい」と言います

でもメピボセテは 不平を言ったり
復讐心に燃えたりせず

自分が平和をつくりたいと
思っていることを示しました

不当な扱いを甘んじて受け入れたのです

メピボセテがダビデに何と言ったか

サムエル第二 19章を見てみましょう

19章の30節です

「メピボセテは王に言った

『王が無事に家に戻られたのですから

全部彼のものになっても構いません』」

そうです メピボセテにとって

ダビデ王が無事に戻ってきたことの方が

自分のことより大事でした

これから流れるビデオの中で

ある兄弟が不当に思える処置に
どう反応するかを見ましょう

そしてその後 どのように平和の種を
まくかに注目しましょう

僕たちは刑務所の中で
デービッドに会ったんだよね

看守だった時に
いろんな人を見てきましたけど

こんな平和な人たちを見たのは
初めてでした

牢屋に閉じ込められても

エホバからの平和はいつもありますよね

平和をつくるために努力したよね

フィル

そうでしたね

実は ちょっとした誤解があって

それが大きな問題になっちゃったんです

私の対応も良くなかった

自分の意見を言えば言うほど

状況が悪くなっていった

これは何かの間違いだと思います

私がこんなことするはず
ないじゃないですか

まさかこんなことになるとは
思わなかった

もう1つお知らせがあります

フィリップ・キム兄弟は
長老として奉仕しておられません

これで今日の集会は終わります
できる方は起立して⋯⋯

その後の数カ月は本当に苦しかった

そのうち詳しい事情が明らかになり

また長老になれると思った

でもなれなかった

フィル!

なぜなのか分からなかった

フィル ちょっと待って

怒りが

頼むから

込み上げてきた

フィル ちょっとだけ・・・

自分のことしか考えられなかった

こんなことが許されるのか と思った

頭が真っ白になって

エホバのことが
見えなくなっていたんです

エホバの助けが本当に必要でした

誤解されてもエホバに忠実に仕え続けた
聖書中の人たちの例を調べた

それを読んで考えさせられた

エホバの組織にいられるだけで

幸せなことなんだと気付かされた

兄弟たちとの平和は

どんな立場で奉仕するか
ということより重要なんです

平和がなかったら その後の変化に
対応できたかどうか分からない

関節 痛むんですね

私が書きますよ

エレミヤの秘書官バルクみたいに

フィルがいてくれてほんと助かるよ

では 私たち皆が考えたいのは

どうすればメピボセテの模範に
倣えるかということです

例えば 何かささいな
うわさを立てられたり

失礼なことをされたりしたら
どうしますか

事を荒立てるのではなく 

気にしないようにできるかもしれません

ペテロ第一 4:8に従えます

その聖句にはこうあります

「何よりも 熱烈に
愛し合ってください」

そして 愛するとどうなるかも
述べられています

「愛は多くの罪を覆うからです」
とあります

そうです 私たちが進んで愛を表すなら

相手の罪や不完全さを
許すことができます

では もっと深刻な状況に置かれた
場合にはどうしたらいいでしょうか

例えば ビデオの中の兄弟は

長老の資格を失った時

自分は不当な扱いを
受けていると感じました

疑いを晴らそうとしましたが

長老たちから思うような反応が
返ってきませんでした

それで感情を害しました

心の平和を失い

結果として他の人たちとの
平和な関係も損なわれてしまいました

兄弟は平和を取り戻すため

聖書をじっくり調べることにし

自分にとって参考になる例を探しました

そしてエホバに祈って助けを求めました

自制することや

神の聖なる力に導かれることの
大切さを思い起こしました

その結果 心の平和を
取り戻すことができ

平和の種をまくことができました

ダビデが下した不公平な
決定について思い返してみましょう

メピボセテに対して

「ツィバと土地を分け合いなさい」
と言いました

悪いのはツィバで メピボセテは
何もしていないのにです

それでもメピボセテは 神の聖なる力に
導かれている人にふさわしい態度を示し 

自分の権利に固執しませんでした

よく辛抱し

復讐心を抱いたりしませんでした

ご一緒にサムエル第二 19章を
読んでみましょう

25節から28節です

サムエル第二 19:25-28

「彼[つまり メピボセテ]が

王を迎えるためにエルサレムに来た時

王は彼にこう言った

『メピボセテ あなたはどうして
私と一緒に来なかったのですか』

メピボセテは言った

『ご主人さま 王よ 私は召し使いの
たくらみにはまったのです

私は足が不自由ですので

「ロバに乗って王と一緒に行けるよう

くらを置いておいてください」
と言ってありました

それなのに 彼は王の前で
私のことを中傷しました

ですが 王は真の神の
天使のような方ですから

王が良いと思うことをなさってください

私の父の家の者は皆

王に滅ぼされてもおかしくないのに

あなたは私を同じ食卓に
着かせてくださいました

私には これ以上何かを王に
求める権利などございません』」

素晴らしいですね

メピボセテはあるもので
満足していました

ほかに何も求めませんでした

それで心の平和を感じていたのです

私たちは 中傷や詐欺の被害を
受けたらどうすることができますか

マタイ 18:15-17のイエスの
指示に従うことにするかもしれません

では その通りに行動したのに問題が
解決されない場合はどうしますか

もうその件について それ以上
追求しないことにするかもしれません

その方が 会衆の平和を乱すより
良い場合があります

解決されない問題は
エホバに委ねることができます

エホバは 私たちの行動も
他の人の行動も見ています

いずれ物事が正されるように
してくださることでしょう

私たちは憤りの気持ちを捨てるように
努力する必要があります

そうすることは私たちのためになります

人を許すなら エホバも私たちのことを
許してくださいます

これは決して 仲間の重大な過ちを
見過ごすということではありません

でも ぜひメピボセテに倣って

憤りの気持ちを捨て

平和を追い求めたいと思います

エホバから離れないように

詩編 55:22の
アドバイスに従いましょう

「重荷をエホバに委ねよ
そうすれば支えてくださる

神は正しい人が倒れることを
決して許さない」

ローマ 15:13に
保証されている通り

エホバが「信仰を持つ皆さんを

あらゆる喜びと平和で
満たして」くださいますように

奉仕委員会の援助者
ジョエル・デリンガー兄弟が

このシンポジウムの
次の話をしてくださいます

主題は 「平和の種をまいて
平和を刈り取った人たち 

パウロとバルナバ」

次に考える問題は何でしょうか

長老同士の口論や仲たがいです

長老の皆さん 経験がありますか

ついかっとなってしまって
強い口調で話したり

関係が悪くなってしまったり
したことがあるでしょうか

長老たちは模範的であることが
期待されていますが

長老たちを含め 私たち皆は
不完全だというのが現実です

生い立ちや性格や考え方などが違うので

誰かとぶつかってしまうことがあります

その状況を放っておくなら 平和が
乱されてしまうことになりかねません

この話では 1世紀に
パウロとバルナバが

どのように平和の種をまいて
平和を刈り取ったかを考えます

2人の例から 現代の長老たちが
仲たがいを解決するのに役立つ事柄を

ぜひ学ぶことにいたしましょう

ご一緒に使徒 15章を開いて
36節から39節を読んでみましょう

パウロとバルナバの間にどんな問題が
生じたかに注目してください

使徒 15章の36節からです

「何日か後 パウロはバルナバに言った

『さあ エホバの言葉を広めた
全ての町に戻って兄弟たちを訪ね

どうしているかを見てみましょう』

バルナバは マルコと呼ばれるヨハネを
連れていくことに決めていた

しかしパウロは パンフリアで

マルコが一緒に行動するのを
やめてしまったことがあるので

彼を連れていくことに賛成できなかった

そこで怒りが激しくぶつかって

2人は別れることになった

バルナバはマルコを連れて
船でキプロスに向かった」

問題は何でしたか

パウロとバルナバは 宣教旅行に
マルコを連れていくかどうかについて

意見が合いませんでした

もちろん 違う意見を言うこと自体は
間違ってはいません

15:7によると

エルサレムの使徒や長老たちは
「活発な論議」を行い

それが良い結果につながりました

でもパウロたちはどうなりましたか

39節を見ましょう
ここに「パウロとバルナバは

エホバに祈って合意に達した」と
書かれていたらよかったですね

でも残念なことに
この時の話し合いでは

パウロとバルナバの
「怒りが激しくぶつかって

2人は別れることに」なりました

2人の友情に一時的に
ひびが入ってしまったようです

でも それは一時的でした

しばらくして パウロとバルナバは
仲直りすることができたようだからです

少し後に パウロは「ガラテアの
クリスチャンへの手紙」の中で

バルナバや一緒にした活動に
ついて書いています

争ったことについては触れていません

さらに コロサイのクリスチャンや
テモテに書いた手紙の中で

パウロはマルコのことを褒めています

最後に書いた「テモテへの第二の
手紙」の中では こう言っています

「マルコを連れてきてください
私の奉仕を支えてくれるからです」

ですから パウロとバルナバは

仲たがいを解決することが
できたようです

では考えてみましょう

どうしてそうできたのでしょうか
これが理由です

仲たがいをするずっと前から
2人は一緒に熱心に宣教をしていました

何年もの間 聖書が言う
「平和という絆」で

2人は結ばれていたのです

ですから 友情に少しひびが入っても

2人の絆が強かったので
すぐに関係を修復することができました

では現代に当てはめましょう

次のビデオの中で ある長老は
どのように平和の種をまくでしょうか

この話には続きがあって
フィルは知らなかったんですが

ティムとの関係が
険悪になっちゃったんです

知りませんでした

もっと穏やかに

話していればよかったんです

ああなる前に

カール

ちょっと話があるんですけど

フィルの資格 再検討しないと
いけないんじゃないですか

フィルはあんな兄弟じゃないって・・・

いや いや もうフィルのことはいい
友達だからそう言っているんだろ・・・

友達?

長老団はフィルには長老としての
資格がないと決めたんだ

友達?友達って 一緒に何年も
働いてきた仲じゃないですか

長老団は全員一致で
資格がないと決めたんだ

カールが圧力をかけたから
じゃないですか!

なんだって?

世の人からの反対がある上に

会衆の中でそんな圧力をかけて
どうするんですか!

本気でそんなことを?

話し合いというより
けんかになって ひどかった

2人とも言い過ぎてしまった

その後の対応も良くなかった

なんとかする必要があった

簡単ではなかったけど

きちんと話し合った

エホバの助けで

仲直りできた

そうしていなかったら
刑務所の中でどうなっていたんだろうね

みんなと同じ監房に入れられてから

さっきのことをフィルに話したんだ

そうだったんですか

歌うよ!

行こうか

この世界がどんよりと暗く沈んでも

信仰の目には見える 明るい未来が

川を越えて山も越え その向こうまで

この地球の果てにまで⋯⋯

では長老の皆さん
次の点を考えましょう

どうすればパウロとバルナバに
倣えるでしょうか

まず 仲間の長老たちに接する時にも
聖書の教えを当てはめる必要があります

ビデオの中で ティムとカールは
激しくぶつかっていましたね

それが身振りや言葉や口調に
表れていました

ティムは「けんかになって
ひどかった」と言っていました

でもけんかの後 2人はやがて
どんな正しいことをしましたか

聖書の教えを当てはめて
仲直りしました

ヤコブ 3:17は 私たち皆が

「平和を求め 分別」を示す
必要があると教えています

この点が以前の「ものみの塔」誌に
次のように説明されていました

「平和を作る人は
慎み深く自分の考えを述べ

敬意を込めて人の考えに
耳を傾けます ⋯⋯

自分のやり方にこだわらず

他の兄弟の見解を
祈りのうちに考慮します

たいていの場合
聖書の原則に反しないで

さまざまな見方を入れる
余地があるものです ⋯⋯

経験のある監督は 平和を保つほうが

自分の思い通りに事を運ぶより
重要であることを心得ています」

当てはめたいですね

では 1つ目の点として

仲間の長老と接する時にも
ぜひ聖書の教えを当てはめましょう

そうすれば 意見が合わなくても
対立してしまうことはないでしょう

ローマ 12:10に
学べる2つ目の点が書かれているので

そこを開いて読んでみましょう

ここで使徒パウロは 神の聖なる力に
導かれて次のように書きました

「兄弟愛を抱いて

優しい愛情を示し合いましょう

自分の方から進んで
人を敬ってください」

長老の皆さん
終わりの時代である今こそ

私たちはこのことを行う必要があります

互いへの愛をますます強くする
必要があるのです

そのために 誰と話していても

仲間の長老をけなしたりせず
褒めるようにしましょう

意見が合わなくても 腹を立てたり
強い言葉を使ったりしないようにします

仲間の長老の良いところや

一緒に奉仕してきた楽しい思い出に
ついて 思い起こすようにしましょう

ビデオの中で ティムとカールが
パウロとバルナバに倣い

平和を刈り取ることができて
本当に良かったですね

簡単ではありませんでしたが

聖書の教えを当てはめて
仲直りできました

カールはティムの家を訪ねた時

笑顔を浮かべていました

お土産はティムの大好きな
パイだったかもしれません

腹を割って話し合いました

そして 刑務所に入った後も
友情が続いていました

ですから 互いへの強い愛を
保っていたことがよく分かります

長老の皆さん 私たちもぜひ
そうありたいものですね

仲たがいをしても
速やかに解決しましょう

そうすれば
平和を刈り取れるだけでなく

パウロとバルナバのように

群れの模範になることが
できるに違いありません

統治体のスティーブン・レット兄弟が

このシンポジウムの
最後の話をしてくださいます

主題は 「平和の種をまいて
平和を刈り取った人たち 現代の例」

世の中の不正を目にしたとき

多くの人はどのように
反応するでしょうか

怒りを感じて
どうにかしたいと思います

それでどうしますか

政治的な活動を行ったり
社会運動に加わったりして

政治の腐敗や社会の不正を
正そうとします

対照的に エホバの証人は
政治活動や社会運動に加わりません

どうしてでしょうか 簡単に言うと

今の世の中の問題は
人間の力では正せないことを

聖書から学んで知っているからです

伝道の書 1:15には

「曲がっているものは真っすぐに
できない」と書かれています

サタンの世の中は 非常に曲がり
くねって節くれ立った木の幹のように

人間の力では決して
真っすぐにすることはできないのです

ですから エホバが正せないと言って
いるさまざまな問題を正そうとして

時間や労力を費やすことには
あまり意味がありません

さらに エホバの証人は
問題だらけの世の中でも

平和や安心感を味わえるということを
学んできました

ではこの点で 現代の2つの例を
考えてみましょう

平和についてのエホバの考えを
学ぶことができた2人の人の例です

その人たちの名前は
エジディオ・ナハクブリアと

フリード・ブルーンです

まずエジディオの例です

東ティモールのへき地で
生まれたエジディオは

子供の頃に悲惨なゲリラ戦を
経験しました

近所の人たちが何人も殺されたので
次は自分かもしれないと思いました

でも しばらくたってから
首都に引っ越して大学に入り

そこで同じような境遇で育った
たくさんの学生と出会いました

エジディオたちは 政治活動をする
学生のグループをつくって

社会改革を目指しました

学生たちで政治デモを
何度も行いましたが

大抵の場合 暴動に発展しました

多くの友人が負傷し
命を落とした人もいます

後に エジディオは親戚と一緒に
エホバの証人から聖書を学び始めました

その時の心境についてこう語っています

「聖書を学ぶにつれ

本当の愛を知らなかったことに
気づきました⋯⋯

わたしの近寄りがたい風貌や
乱暴な気質にもかかわらず

証人たちは⋯⋯『兄弟の愛情』を
示してくれました」

間もなくエジディオは
バプテスマを受けました

そして開拓者になって
多くの人たちを助けるようになりました

ライフ・ストーリーの結びで
次のように述べています

「以前のわたしは怒りに満ち

自分は愛されていない 愛される
価値もないと感じていました

でも エホバのおかげで 真の愛と
安らぎを見いだすことができました」

次はフリードの例です

フリードは幼い頃から 世の中の
憎しみや争いに心を痛めていました

特に 宗教が行ってきたことに

とても当惑していました

憎しみや争いが多くの場合
宗教によって助長され 支持され

少なくとも大目に見られてきたからです

1955年 フリードは
エホバの証人の大会に行きました

その後 エホバの証人と
聖書を学ぶようになりました

いろいろな良いことを学びましたが

特に心を動かされたことがありました

それは 「大娼婦」つまり大いなる
バビロンの実体を知ったことです

その大娼婦は
啓示 17, 18章に出てきます

その大娼婦が表しているのは 世界を
惑わしている間違った宗教全体です

説明されてとりわけ印象に残ったのは
啓示 18:24でした

そこには 「彼女
[つまり大娼婦]の中には ⋯⋯

地上で殺された全ての人の血が
見いだされました」と書かれています

それでフリードは 間違った宗教は

決して国際的な平和を
もたらせないことを理解しました

そうした宗教は政治を支持し
分裂していて 戦争をするからです

対照的に エホバの証人は
国を超えた家族のような関係で

イエスの教え通りに
行動していることが分かりました

フリードは急速に進歩しました

バプテスマを受け 開拓者になり
ギレアデ学校に行きました

そして 長年グアテマラで奉仕し
多くの人たちを助けてきました

本当の平和をどうすれば見いだせるかを
教えてきたのです

では エジディオとフリードの例から
私たちはどんなことを学べるでしょうか

主に3つの点を考えましょう
1つ目です

政治の腐敗や社会の不正は

政治活動や社会運動では

決して正すことができない という
ことを覚えておきましょう

こんな例えを聞いたことがあります

ある人が必死にクモの巣を
取り除こうとしますが

クモを退治することができません

クモの巣を取っても
すぐにまたクモが作るという繰り返しで

問題が解決しません

サタンはいわば この例えの中に
出てくるクモのようです

サタンが腐敗や不正の根本原因なので

サタンがいる限り問題は解決しません

人間にはサタンを退治できません
ずっと強いからです

解決には 人間より
ずっと強い力が必要です

その力を持っているのが神の王国です

神の王国は クモの巣のような
問題を取り除くだけでなく

クモであるサタン自身も
取り除くことができます

では2つ目の点です

政治デモなどは多くの場合
暴力的な活動に発展します

暴力は平和とは正反対のものです

私たちが追い求めるのは平和であって

暴力などではありません

3つ目の点です
ほかのいろいろな宗教とは違って

エホバの証人はあらゆる人と
平和な関係でいることを望んでいます

ですから サタンの世の中の
さまざまな争いで

どちらかの側に付くことは
決してしません

ある若い兄弟の伝道での経験を
思い出します

ある女性が感情的にこう言いました

「あなたが自由を味わえるのも

私の息子が戦争で戦って死んだからよ

あなたは国のために戦わないの?」

兄弟は穏やかにこう言いました

「息子さんが亡くなられたことは
本当に残念です

でも1つ言えることがあります

あなたの息子さんを殺したのは

絶対にエホバの証人ではありません

ほかの宗教はどうか分かりませんが
エホバの証人は人を殺しません」

カトリックだったその女性は

エホバの証人が中立であることを
理解したことでしょう

では 今考えた3つの大切な点を
覚えておくようにしましょう

この大会を視聴している
聖書に関心がある皆さん

もしまだ聖書レッスンをしていないなら

ぜひエホバの証人と聖書を
学んでみることを心からお勧めします

本当の愛を表すことが最高の
生き方であることが分かるでしょう

社会を分裂させているさまざまな
問題も 愛があれば乗り越えられます

イエスがはっきり述べた通り

真のクリスチャンを
見分けるしるしは 愛です

ヨハネ 13:35には

「あなたたちの間に愛があれば
全ての人は

あなたたちが私の弟子であることを
知ります」とあります

聖書に関心がある皆さん
この本物の愛が

エホバの証人の間に見られるかどうか
ぜひ確かめてください

すでにエホバの証人である皆さんは
何ができるでしょうか

ぜひ「信仰で結ばれた兄弟たち
全員を愛」するように

引き続き努力を払ってください

そのためには 世の中の政治的な事柄や
他のさまざまな問題において

いつも中立の立場を取る必要があります

イエスがいつも中立だったようにです

ヨハネ 17:14で
イエスは弟子たちについて

「私が世の人々のようでは
ないのと同じように

彼らも世の人々のようでは
ない」と言いました

では 伝道で人と話す時であれ
ほかのどんな時であれ

政治的な事柄については

いつも中立を保つようにしましょう

こういう法律を作った方がいい
なくした方がいい

変えた方がいい などと
言ってはなりません

自分の見方を他の人に
押し付けるべきでもありません

中立を保つために大事なこととして

自分の考えがメディアの影響を
受けないように注意しましょう

ニュース報道には偏った見方が
表れていることが少なくありません

ですから うのみにするべきでは
ありません

考え方の面で中立であるなら

言葉や行動においても
中立であることができます

もし皆さんがクリスチャンとして
中立の立場を貫くなら

誠実な人たちが真理に
引き寄せられるかもしれません

私たちはこのシンポジウムを通して
どんなことを学べたでしょうか

それぞれの部分で学んだ要点を
1つずつ思い起こしてみましょう

1つ目の話は「ヨセフと兄たち」でした

要点は 憤りの気持ちを捨て去ることに
よって平和の種をまくことができます

2つ目の話は「ギベオンの人たち」

自分の望んでいることと違ったとしても

謙遜にエホバの指示に
従うことが大切です

3つ目は「ギデオン」

不和が深刻な問題に発展しないように

よく考えられた快い言葉を
語るように努力できます

4つ目は「アビガイル」

平和をつくるためにベストを尽くして

後はエホバに委ねることにしましょう

5つ目は「メピボセテ」

被害を受けたままにする方が

会衆の平和を乱すより
良い場合があります

この点に少し付け加えると

自分の名誉よりも エホバの評判を
守ることの方がはるかに大事です

そのことをいつも
忘れないようにしましょう

ですから 自分のことよりも

会衆の平和を守る方が
よい場合があるのです

そして この話は「現代の例」でした

神の民と共に平和の神に仕えるなら

平和を味わうことができます

このシンポジウムを通して
どうすれば平和の種をまけるか

多くの点を学べました

ぜひ平和の種をまき続けましょう

そうするとどうなるでしょうか

答えは このシンポジウムの
最後のビデオを見ると分かります

ご一緒に見ましょう

私も話したいことがあります

まだ気付いていなかったのですが

私の心に真理の種がまかれていたんです

皆さんは明らかに違っていました

なぜだろう と思いました

刑務所にいるのに

皆さん自由に見えました

そんなのは初めて見ました

皆さんが釈放されてから

皆さんには神の祝福があったんだ
ということがはっきり分かりました

その違いはなんだろう と思いました

フィルが答えを教えてくれました

「真理を知り

真理によって自由になります」

フィルは 真理に従って生きるよう
勧めてくれました

最初は難しく感じました

世界がひどく混乱しているのに

平和を貫き神の王国を待つ
という感覚が

どうしても分かりませんでした

でも エホバの助けと

皆さんの辛抱のおかげで

大患難が始まる前に

バプテスマを受けることができました

ヨセフ 私たちは投獄された時

牢屋に入れられても
エホバがいつも共にいたという

あなたの話から
本当に力をもらっていたんですよ

ある時カール 言ってましたよね

なんでしたっけ?

「いつかヨセフを呼んで食事会をして

みんなで『ありがとう』って言おう」

実現してうれしいです

いや 感謝したいのは私の方です

皆さんのそれぞれの話を聞いて

感動しました

皆さん 私のこと
よく知っていると思いますけど

もう少しお話ししてもいいですか

ぜひ!たくさん聞きたいことあります

エジプトのこととか

ファラオどんな人でした?

あ 夢!夢のことも聞かなくっちゃ

ほんとにたくさん質問あるんですね

いくらでも喜んでお話ししますよ

神エホバに感謝すべきことに

時間はたっぷりありますからね

感動的なビデオでした

一生懸命に平和の種をまくなら

私たちは自分自身や他の人たちを
救うことになります

このシンポジウムを通して

聖書のどんな基本的な真理について
学べたでしょうか

それはガラテア 6:7に
書かれていることです

「人は自分がまいているものを

必ず刈り取ることになります」

例えば トマトの種をまいたら

必ずトマトを刈り取れます

この基本的な真理は
平和にも当てはまります

平和の種をまくなら
必ず平和を刈り取れます

もちろん 平和の種をまいてから

平和を刈り取るまでに
時間がかかる場合もあるでしょう

でも粘り強く平和の種をまくなら

私たちは心の平和や

他の人たちとの平和 そして
エホバとの平和な関係を刈り取れます

また 永遠に生きるという
報いも刈り取り

平和に満ちた世界で
いつまでも幸せに暮らせます

平和の種をまくべきなのは今です

そうすれば 将来ずっと
平和を刈り取れるのです

時間や労力を費やす
価値があると言えますね

では 今平和の種をまいて

永遠にわたって平和を刈り取りましょう

兄弟たち ありがとうございました

話で取り上げられた良い手本に倣い

今平和の種をまいて 永遠にわたって
平和を刈り取りたいと思います

では 28番の歌を歌いましょう

主題は「エホバの友となる」です
歌は28番です

招待に応じてこの大会を
視聴してくださっている皆さんを

温かく歓迎いたします

今の世の中で本当の友達を
見つけるのは簡単ではありません

では 神と親しくなって友達に
なることはできるのでしょうか

もしできるとしたら
何をすればよいのでしょうか

統治体の成員 ケニス・クック兄弟が

講演の中でそうした点に
答えてくださいます

主題は 「神と友達になるには」です

神と友達になることなんて
できないと思いますか

そう思っている皆さんに

ぜひ知っていただきたい点があります

今 何百万人もの人たちは

実際に神と友達になっています

その人たちは平和で充実した毎日を送り

いつまでも平和に暮らせる
という希望を持っています

自分もそうなりたいと思うなら

ぜひ聖書に基づくこの話を
聞いてください

なぜ聞いていただきたいかというと

多くの人は神の友達になれるのか
という重要な質問について

考えようとしないからです

多くの人がこのことを
気に留めないのはなぜでしょうか

理由はさまざまです

ある人たちは 神なんて
遠い存在だと考えています

神はいないとか 人間には
関心がないと言う人もいます

神は謎めいていて 残酷で 厳しくて

要求ばかりすると言う人もいます

多くの宗教では 神が悪人を永遠の
責め苦に遭わせると教えています

そのような神と友達に
なりたいと思いますか

とてもそうは思えないでしょう

ある人たちは 世の中の
不公正や苦しみを見て

神は死んでしまったに違いないとか

初めからいなかったんだと言います

ある人たちは 自分は悪いことを
たくさんしてしまったので

神の友達になんてなれないと
思っています

あなたもそんなふうに
思ったことがありますか

もしそうなら この話は
きっと役立つでしょう

4つの点を考えます

自分も神と友達になれると
思えるかもしれません

では考えていきましょう

1つ目に 私たちを神から
引き離したのは罪です

2つ目に 神は人間のために
自分の方から行動し

良い関係を取り戻せるように
してくださいました

3つ目に いつまでも
神の友達でいるためには

しなければならないことがあります

4つ目に 神の友達になる人は

今も将来も平和を楽しめます

では1つ目の点を考えましょう

私たちを神から引き離したのは
何でしょうか

一言で言うと罪です

なぜそう言えますか

なぜなら 真の神は清く聖なる方で

行いが完全だからです

罪が全くなく 最高度に清い方です

人間は誰一人 神と同じほど
聖なる状態になることはできません

「人間」と「不完全」という言葉に
ついて考えましょう

人間といえば不完全
不完全といえば人間です

そうではないでしょうか

人間の歴史を見ると

その現実が痛いほどよく分かります

罪や不完全さによって壁ができました

私たちと神との間にです

私たちは神から
引き離されてしまったのです

その点はイザヤ 59:2に
書かれています

「あなたたちは自らの過ちによって
神から引き離された⋯⋯

あなたたちの罪のせいで
神は顔を隠した」

「神から引き離された」とあります

生まれつきこういう状態です

でも私たちが罪人なのは

人間を造った神のせいではありません

最初の人アダムとエバは完全でした

でも神への感謝を忘れて自分勝手になり

神との友情を失いました

神に罪を犯したので 神と
平和な関係でいられなくなり

敵になりました

2人から生まれた子孫は

罪と死を受け継ぎました

どれほどひどい結果になったかが

ローマ 5:12に書かれています

ここで使徒パウロは
罪と死がどう関係しているのかを

短い言葉で説明しています
こうあります

「このような訳で 1人の人によって
人類に罪が入り

罪によって死が入り

こうして全ての人が罪人になったために

死が全ての人に広がった」

「死が全ての人に広が」りました

この罪というパンデミックは

感染率が100パーセントです

死亡率も100パーセントです

誰も逃れられません

「全ての人が罪人になったために

死が⋯⋯広がった」とあります

自力ではこの状態から
抜け出すことができません

深い穴の中に落ちて

出口が見えないようなものです

でも希望がないわけではありません

神は全ての人を敵と見ている
わけではないのです

ご自分の友と見ている人もいます

例えば 「ヤコブの手紙」 2:23は

アブラハムという人について
こう言っています

「『アブラハムはエホバに信仰を持ち

そのことは正しいと見なされた』⋯⋯

アブラハムはエホバの友と
呼ばれるようになったのです」

アブラハムは私たちと同様
不完全でした

でも 神の友と呼ばれる
ようになりました

不完全な人間がどうして神の友達に
なることができるのでしょうか

簡単に言うと 神の方から
行動してくださいました

私たちとの良い関係を
取り戻すために行動し

私たちが神に近づけるように
してくださったのです

どのようにでしょうか

まず聖書によると

神は私たちがご自分を知り

ご自分との友情を育むことを
望んでいます

実際 そうするよう勧めています

使徒 17:27によると

神は私たちが神を誠実に
知ろうとすることを願っています

実際「神は 私たち一人一人から
遠く離れてはいません」

次はとても大切です

神は自分の方から

子であるイエスを犠牲として
与えてくれました

私たちを罪から自由にするためです

神はこのようにして
最大の愛を示してくださいました

ヨハネ第一 4:10を読みながら

神がどれほど大きな愛を示してくれたか
について考えてみてください

私たちとの間にできた壁を取り除くため

自分の方から行動してくれたのです

ヨハネ第一 4:10

「私たちが神を愛したというより

神が私たちを愛し

私たちの罪を償う犠牲として
ご自分の子を遣わしてくださったのです

これこそが愛です」

「神[は]私たちを愛し⋯⋯

ご自分の子を遣わしてくださ」いました

神は犠牲としてご自分の子を
遣わしました

その犠牲には罪を「償う」力があります

どういうことでしょうか

アダムは完全な命を失いました

イエスが犠牲として差し出したのも

人間としての完全な命です

アダムは神に反逆してしまったため

完全な命を失いました

その結果 アダムの子孫は

罪を代々受け継ぎ

神から引き離され
神との友情を失ったのです

贖いと呼ばれるイエスの犠牲には

アダムが失ったものを
取り戻す力があります

神の公正という基準に
かなっているからです

神は大きな犠牲を払って
この贈り物をしてくれました

神は私たちにご自分のことを
知ってほしい

友達になってほしいと願っているのです

私たちが神と親しくなり
友達になれるよう

神の方から行動して

道を開いてくれたことを
うれしく思いませんか

3つ目です
私たちには何が求められますか

謙遜さです 神からの贈り物を
謙遜に受け取る必要があります

謙遜であれば 神の憐れみが
必要であることを認め

感謝を表すために
精いっぱいのことを行うはずです

「使徒の活動」 3:19にある通り

「悔い改めて生き方を
変え」なければなりません

ですから 憐れみを含む
神からの贈り物を頂くには

悔い改めることによって考え方を変え

行いや生き方の点で

必要な変化を遂げなければ
ならないのです

そうする機会は
全ての人に開かれています

創造者である神は
全ての人に憐れみを示しています

友達になりたいと思っています

そのために必要なことを
全部してくれました

私たちがすべきなのは
感謝して 謙遜になり

次のことを行うことです

イエスとイエスの教えを
受け入れることです

神はイエスの手本に倣うようにと
言っています

どの程度倣うべきでしょうか

ヨハネ 14:6でイエスは
こう言っています

「私は道であり 真理であり 命です

私を通してでなければ 誰も
父のもとに行くことはできません」

ですから 神の友達になるには

イエスにしっかり従う必要があります

ここまでで 人間が罪によって
神から引き離されたこと

また私たちが神の友達になれるよう

神がどのように壁を取り除いて
くださったかを考えました

では次の点を考えましょう

私たちは神の友達になるために

何をしなければならないでしょうか

ご一緒にヤコブ 4:8を開いて
考えてみましょう

聖書をお持ちであれば
どうぞ目で追ってください

ヤコブ 4:8です

次の点に注目してください

どんなことが呼び掛けられているか

神は何をしてくれるか

私たちは何をすべきか です
こうあります

「神に近づいてください そうすれば
神は近づいてくださいます

罪人たち 手を清めてください

優柔不断な人たち
心を清めてください」

まず「神に近づいてください」と
呼び掛けられています

神に近づくなら
神は何をしてくれるでしょうか

「近づいてくださいます」

でも神に近づいてもらうため

しなければならないことがあります

神の基準に従って

自分を清めなければなりません

そのためには努力が必要です

神の基準を学ばなければなりません

その基準が一番
良いものであることを信じ

それに従うことを決意
しなければなりません

神に喜ばれたい 神の友達になりたい
という気持ちからそうします

では どうすれば神との友情を
育てられるでしょうか

進んで行動し努力することです

植物を育てているとしましょう

植物には世話が必要です

定期的に水をやり 成長に適した環境を
整えなければなりません

神との友情を育てるためにも
同じようなことが必要です

聖書を学ぶことによって
神との友情を育てることができます

聖書から神について
大切なことを学べます

神の名がエホバであることが分かります

私たちは友達を名前で呼びます

ですから 神の名を使うことは大切です

ヤコブ 4:8に書かれている通り

私たちが神に「近づ」くために
誠実に努力するなら

エホバは「近づ」いてくださいます

エホバは 決して
真の友を見捨てたりしません

ヨハネ 17:3にある通り

神のことを知るなら

「永遠の命を得る」ことができます

これはエホバが真の友に
惜しみなく与える

貴重な贈り物です

あなたも神の友達になって
幸せに暮らしたいと思いませんか

そうであれば ぜひエホバの証人と
無料の聖書レッスンを楽しんでください

テキストは「いつまでも幸せに
暮らせます」という出版物です

これです

エホバの証人から印刷版や電子版を
入手することができます

最初の3つのレッスンは
「聖書はどのように役立つ?」

「聖書を読むと希望が持てる」

「聖書は信用できる本?」です

こうした点を学び

書かれている真理に沿って
神を崇拝するようお勧めします

ここまでで 神の友達になるには
どうしたらよいかについて考えました

まず聖書を学んで
神について知ることです

イエスは そうすれば
「永遠の命を得」られると言いました

そして神の名を知り 使うことです

神の友達になりたいと願う人が
しなければならないことは

ほかにもあります

神の性格に倣うこと

神が嫌うことを考えたり
行ったりしないこと

神の友達になりたい人と交友を持つこと

神の基準を無視したり
ばかにしたりする人と

付き合わないようにすることです

今挙げた中に 神の友達に
なりたい人との交友を持ち

神の基準を無視したり
ばかにしたりする人と

付き合わないように
することがありました

これは当然ではないでしょうか

先ほどの植物の例えを
思い出してください

水や肥料をやるだけでなく

植物が成長しやすい環境を
整えなければなりません

私たちにも良い環境が必要です

そのためには 神の友になりたい人を
友にしなければなりません

では 神の性格に倣うことについては
どうでしょうか

不完全な私たちにできるのでしょうか

神の性格の2つの特徴を取り上げ

どのように倣えるかを考えましょう

まず愛について考えます

神のような愛を身に着け
神の友達になるには

イエスが教えたことを
実践しなければなりません

例えばイエスはこう教えました

「人からしてほしいと思う通りに
人にもしなさい」

シンプルな教えですが
実践するには努力が要ります

でも不可能なことではありません

神の友達になりたいと思う人は

そのために一生懸命努力します

神の愛に倣うには 神のような愛情深い
まなざしで人々を見る必要があります

相手の身になって考え
何を必要とし 何を心配し

どんな心の痛みを抱えているかを
理解するよう努めましょう

けがや病気や老化のために

つらい思いをしているかもしれません

気分が落ち込んだり 心配事を
抱えたりしている人もいることでしょう

その人にしか分からないような
問題もあります

例えば 若い人には高齢の人の気持ちが
なかなか分からないかもしれません

ちょっと考えただけでは
年を取るというのが

どういうことか分からないでしょう

では どうすれば神のような愛を
身に着けられるでしょうか

高齢の人の気持ちがある程度
理解できるようになるまで

話によく耳を傾けることです

そうすれば 神のような愛を
少しずつ身に着けられます

エホバの友達になりたいと
思っている人は

エホバの愛に倣い 相手が本当に
必要としている事柄を考えて行動します

では 神の性格の2つ目の特徴である
親切に注目しましょう

神の友達になりたいと思う人は

人に親切でなければなりません

この点についてもイエスの
言動から多くを学べます

イエスは至高者である神が

「感謝しない悪人にも
親切」だと言いました

イエスは天のお父さんエホバの
親切に感動し

同じような親切を示しました

どのようにそうしたでしょうか

自分がこういうことを言ったら
あるいはこういうことをしたら

相手がどんな気持ちになるだろう と
よく考えたのです

ある時 罪人として知られる女性が

イエスの所に来て 泣いて

涙でイエスの足をぬらし始めました

イエスは彼女が悔い改めていることを
見て取りました

そして不親切にあしらったら
とても傷つくに違いないと思いました

天のお父さんと同じように

この女性の良いところを見たのです

そして彼女を褒め

彼女の罪を許しました

私たちもどうすれば
神の親切に倣えるでしょうか

いつでも誰にでも

優しく接することによってです

親切な人は 相手の気持ちを
察することができます

相手の感情を傷つけないよう
気を付けます

自分がこういうことを言ったら

こういうことをしたら相手は
どう思うだろうと考えるのです

神の友達である人は
人にそのように接します

ご一緒にエホバの愛と
親切について考えました

このように聖書を読んで
学んだことを当てはめることは

本当に私たちのためになります

神と友達になることができます

神の素晴らしい性格に
近づくこともできます

次のビデオをご覧ください

聖書を使ってどのように神との友情を
育てることができるでしょうか

神様にこう祈ったことがありますか

「もしおられるなら
どこにおられるのですか

私のことを気に掛けてくれていますか」

昔から 世界中の多くの人が

人間を造った神について
知りたいと思ってきました

父親のことをよく知らない
女性がいたとします

「あなたはお父さんに捨てられたのよ」
と言われて育ってきました

でも 心の中では「それは違う」
と感じていました

ある日 父親から手紙が来ました

手紙を読んで お父さんは生きていて

自分が生まれた時からずっと気に掛け

助けになりたいと思ってくれていた
ことが分かりました

そして お父さんがどこにいるかも
分かりました

その後

会うことができました

お互いのことをよく知って

親友のような強い絆が生まれました

聖書は お父さんともいえる
神からの手紙のようなものです

聖書を読むと

神にどのように祈れるか

神はどのように答えてくれるか

私たちのことをどれほど気に掛けて
くれているかが分かります

また 考えたこともないような

素晴らしいことが可能になる
と言っています

神と友達になれるのです

神と友達になるには

まず名前を知る必要があります

神はその名前を教えてくれています

「エホバという名を持つあなただけが

地球全体を治める至高者である」
と言っています

そして こう約束してくれています

「神に近づいてください

そうすれば
神は近づいてくださいます」

優しいお父さんエホバは

神を知ろうとする人を
喜んで友達にしてくれます

あなたは神の友達になりますか

そのことや ほかの点について
もっと知りたい方は

jw.orgをご覧ください

とても興味深いビデオでした

あなたも神について
知りたいと思いますか

あなたも呼び掛けに応じ

神の友達になるために
頑張ってみたいと思いますか

聖書には そのために必要な
アドバイスが書かれています

聖書のアドバイスに従うなら

将来 幸せになれます

それだけではありません

今も幸せになれます

エホバの友達になるなら今

平和で幸せな生活ができるのです

神の友達になるなら

神はあなたの祈りを聞き
憐れみを示してくれます

罪が許されていることを知ると

深い喜びを味わえます

詩編 32:1, 2にはこうあります

「違反を許され
罪を覆われる人は幸せだ

エホバから罪があると
見なされない人⋯⋯は幸せだ」

エホバに許してもらえるのは
本当にうれしいことです

以前にしてしまったことのために
罪悪感を抱いているとしても

心配は要りません

神を知り 神の友達になる前に
してしまったことは許されます

あなたも神に許してもらうことが
できるのです

ビデオにあったように 聖書を学ぶと
神にどのように祈れるか

祈りがどのように聞かれるかが
分かります

また 神の許しや助けを
求める祈りを含め

神が祈りを聞いてくれることを
確信できるでしょう

格言 15:29には

エホバは「正しい人の
祈りを聞く」とあります

正しい人とは

エホバが求めていることに従って
生きようと努力する人のことです

神の友達になれるのは
本当にうれしいことです

神は私たちの努力に目を留め

祈りを聞いてくださるのです

それだけではありません

神の友達になるなら
他の人との平和な関係も楽しめます

イエスも他の人と平和な関係で
いるよう勧めました

敵でさえ愛するよう教え

お返しを期待せずに
善を行うようにと言ったのです

また ルカ 6:36では
こう言っています

「天の父が憐れみ深いように
憐れみ深くありなさい」

そのように努力するなら

人々は心を動かされ

自分も神の友達になりたいと
思うでしょう

ホセという男性は
神の友となっている人たちと接し

神と人を愛するようになりました

ホセは13歳の時 ゲリラ活動に
加わるようになりました

世の中に見られる不公正を嫌い

その原因になっていると思えた人たちを

憎むようになりました

皆殺しにしようと思いました

そして戦いで多くの仲間が命を落とすと

ますます怒りと復讐心に
燃えるようになりました

手りゅう弾を作りながらこう考えました

「なぜこんなに苦しみがあるんだろう

神はどうして何もしてくれないんだ?」

ホセは何度も涙を流しました

混乱し 失望しました

ホセはやがて 地元の会衆の
エホバの証人と出会いました

初めて行った集会で 何て愛にあふれた
雰囲気なんだろうと思いました

みんなが温かく歓迎してくれました

しばらくして 神が悪の存在を
許しているのはなぜかについて学び

ずっと抱いていた疑問の
答えが分かりました

聖書の知識が深まるにつれ

考え方や生き方が変わりました

神のような愛を身に着けるよう努力し

神の友達になりました

でも 昔の仲間と関係を断つのは
大変でした

王国会館に行くたびに

昔の仲間が後をつけてきました

中には集会に出席した人もいます

ホセがどうしてそんなに変わって
しまったのか知りたかったのです

やがて ホセが危険な存在ではない
ことが分かり 干渉しなくなりました

ホセは17歳の時
バプテスマを受けました

神の友達になったホセは

全時間 伝道するようになりました

人を殺そうとするのではなく

愛と希望にあふれたメッセージを
伝えるようになったのです

エホバの友達になるなら

本当に良い結果になります

以前は敵だった人たちも
互いに仲良くなり

神の友達になることができます

まとめてみましょう

まず 神の友達になれるのはなぜか
ということを考えました

聖書に書かれている通り

神は昔 何人かの人を友と呼びました

神が私たちとの良い関係を取り戻すため

何をしてくれたかも学びました

神は子であるイエスを
遣わしてくれたのです

エホバ神はイエスを
贖いの犠牲として与え

私たちを罪から解放してくれました

また 神の憐れみを受け
神の友達になるために

神の言葉 聖書を学び

神の性格に倣う必要が
あることも学びました

そして どうすれば今幸せな生活を送り

他の人と平和な関係を
持てるかも考えました

エホバはご自分の友が平和な世界で
永遠に生きられるようにしてくれます

あなたもそこで生活したいと
思われませんか

もしそうなら ぜひエホバの証人と
聖書を学ぶようお勧めします

聖書には 苦しみや悲しみのない世界が

いつまでも続くことが約束されています

神はご自分の友の顔から涙を拭い去り

いつまでも幸せに暮らせるように
してくれるのです

これは単なる夢ではありません

神の約束は必ず実現するからです

ありがとうございました
神と友達になるなら

平和に暮らせることが分かりました

今 平和を愛する多くの人が

神と聖書に信仰を
持つようになっています

お望みであれば 聖書を
無料で学ぶことができます

時間と場所はご都合に合わせられます

エホバの証人にお尋ねになるか

jw.orgから聖書レッスンを
お申し込みください

お近くのエホバの証人が
喜んでお手伝いします

言語によっては jw.orgに

自習形式のオンライン聖書講座も
用意されています

これで 3日目午前の部を終了します

次の部ではドラマを
楽しむことができます

「エホバは平和へと導いてくださる」
というドラマの第2部です

その後 大会最後の話をお聞きします

では 神の約束に対する
確信を強める歌を歌いましょう

ご一緒に歌うのは147番の歌です

主題は「約束された永遠の命」です

歌は147番です

その後 それぞれの場所で
祈りを捧げてください
\end{document}
